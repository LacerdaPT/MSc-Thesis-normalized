\begin{resumo}
Para compreender como os neurónios no cérebro produz a diversidade de comportamentos observados, é importante ter à nossa disposição métodos quantitativos e objectivos para medir a actividade neuronal, em particular ao nível das populações neuronais. Desenvolvimentos no design de circuito integrados e em microfabrição permitiu a produção de sondas de muito eléctrodos densas e de grandes dimensões com centenas de eléctrodos. No entanto, os métodos computationais para analisar os dados recolhidos com estas sondas de nova geração não acompanharam a evolução da tecnologia. Segundo Neto et al. (2016), os autores adquiriram dados simultâneos de sondas de vidro com 128 canais e uma pipeta juxtacelular. Neste documento está relatada a aplicação de métodos de aprendizagem usando redes neuronais artificiais profundas com o objectivo de fazer detecção de Potenciais de Acção Extracelulares nos acima mencionados. Foram testadas várias arquitecturas e configurações e aplicou-se a rede a diversos dados do conjunto. Os resultados são comparados com os dados obtido com o SpikeDetekt, método de detecção de Potenciais de Acção proposto por Rossant et al. (2016)
\end{resumo}
