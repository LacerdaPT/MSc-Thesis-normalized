\begin{abstract}

To understand how the brain produces the diversity of behaviours observed in animals it is important to have quantitative methods to measure neural activity, particularly at the population level. Recent developments in integrated circuit design and microfabrication have made possible the production of large and dense multi-electrode arrays with hundreds of electrodes. However, computational methods to analyze the data recorded by these new generation probes did not keep up with the technological evolution. In Neto et al. (2016), a ground-truth dataset from simultaneous in-vivo recordings of 128-channel dense extracellular silicon probe and a juxtacellular probe was first presented. In this work, these data were used to evaluate the performance and limitations of a recent method for spike detection proposed in Rossant et al. called SpikeDetekt. After, is reported the application of methods for deep learning to perform detection of Extracellular Action Potentials from the same data. Different models are tested and applied to same data. In all datasets the performances achieved were equal or better than those achieved with SpikeDetekt. I conclude discussing limitations of this new method and proposing possible directions for improvement.

\end{abstract}
