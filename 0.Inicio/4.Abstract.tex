\begin{abstract}

To understand how neurons produces the diversity of behaviours observed in animals it is important to have quantitative methods to measure neural activity, particularly on the population level. Developments in integrated circuit design and microfabrication have made possible the production of large and dense multi-electrode arrays with hundreds of electrodes. However, the computational methods to analyse the data recorded by these new generation probes didn't keep up with the technological evolution. In Neto et al. (2016), they obtained a dataset of simultaneous recording of 128-channel silicon probe and a juxtacellular probe. In this document is report the application of method of deep learning to perform detection of Extracellular Action Potentials from this "ground-truth" data. Different models are tested and applied to several recording in the dataset. The results are compared with a recent method from Rossant et al. (2016) called SpikeDetekt.

\end{abstract}
