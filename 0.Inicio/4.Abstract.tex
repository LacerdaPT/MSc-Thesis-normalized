\begin{abstract}

To understand how the brain produces the diversity of behaviours observed in animals it is important to have quantitative methods to measure neural activity, particularly at the population level. Recent developments in integrated circuit design and microfabrication have made possible the production of large and dense multi-electrode arrays with hundreds of electrodes. However, computational methods to analyse the data recorded by these new generation probes did not keep up with the technological evolution. In Neto et al. (2016), a dataset from simultaneous in-vivo recordings of 128-channel dense extracellular silicon probe and a juxtacellular probe was first presented. In this work is reported the application of methods for deep learning to perform detection of Extracellular Action Potentials from this "ground-truth" data. Different models are tested and applied to several recordings in the dataset. The results are compared with a recent method from Rossant et al. (2016) called SpikeDetekt.

\end{abstract}
